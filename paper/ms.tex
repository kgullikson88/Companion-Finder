%\documentclass[12pt,preprint]{aastex}
\documentclass[twocolumn]{emulateapj}
\usepackage{graphicx}
\usepackage{natbib}
\usepackage{subfigure}
\usepackage{url}
%\usepackage{algpseudocode}
\usepackage{verbatim}
\usepackage{ amssymb }
\bibliographystyle{astroads}
\providecommand{\e}[1]{\ensuremath{\times 10^{#1}}}
\begin{document}

\title{Mining Planet Search Data for Binary Stars: The $\psi^1$ Draconis system [UPDATE PLOTS!]}
\author{Kevin Gullikson \altaffilmark{1}}
%\affil{University of Texas, Department of Astronomy \email{kgulliks@astro.as.utexas.edu}}
\author{Michael Endl \altaffilmark{1}}
\author{William D. Cochran \altaffilmark{1}}
\author{Phillip J. MacQueen \altaffilmark{1}}

\altaffiltext{1}{University of Texas}



\begin{abstract}
Several planet-search groups have acquired a great deal of data in the form of time-series spectra of several hundred nearby stars with time baselines of over a decade. While binary star detections are generally not the goal of these long-term monitoring efforts, the binary stars hiding in existing planet search data are precisely the type that are too close to the primary star to detect with imaging or interferometry techniques. We use a cross-correlation analysis to detect the spectral lines of a new low-mass companion to $\psi^1$ Draconis A, which has a known roughly equal-mass companion at $\sim 680$ AU. We measure the mass of $\psi^1$ Draconis C as $M_2 = 0.70 \pm 0.07 M_{\odot}$, with an orbital period of $\sim 20$ years. This technique could be used to characterize binary companions to many stars that show large-amplitude modulation or linear trends in radial velocity data.
\end{abstract}

\maketitle

\section{Introduction}
\label{sec:intro}
Several groups \citep[e.g.][]{Wittenmyer2006, Fischer2009, Pepe2011} have used the radial velocity method to search for planets around nearby stars for well over a decade, and have collectively uncovered several hundred planets to date. Close binary stars are usually cut from the star sample because they complicate the detection method \citep[e.g.][]{Bergmann2015}, and because they have long been suspected to inhibit planet formation by quickly destroying \citep{Kraus2012} or depleting \citep{Harris2012} the planet-forming disk.

Previously unknown stellar binary companions are nonetheless still uncovered in planet search data through large amplitude linear trends or even full long period orbits, but may be ignored since the goal is to find planet mass companions. Since binary stars are usually excluded in the star sample, companions that are found tend to have extreme flux- and mass-ratios. Binary stars with extreme mass-ratios on orbits with $\sim 10$ year timescales are precisely the ones that are most difficult to detect and characterize with imaging techniques, and so they should not be ignored.

Several groups have recently worked towards using high-resolution spectroscopy to search for very faint companions to nearby stars, both in the context of detecting emission \citep{Snellen2010, Gullikson2013} or reflection \citep{Martins2013} from ''Hot Jupiter'' planets, and in the context of detecting stellar binary systems with high contrast ratios \citep[e.g.][]{Gullikson2013_2, Kolbl2015}. These groups all use a cross-correlation analysis to search directly for the spectral lines of the faint companion and mainly differ in their treatment of the primary star and telluric lines.

In this paper we use data from the McDonald Observatory Planet Search team to examine the $\psi^1$ Draconis system, which consists of an F5IV-V star ($\psi^1$ Dra A) orbited by a G0V star ($\psi^1$ Dra B) with angular separation $30''$\citep{WDS}. \cite{Tokovinin2002} searched for signs of a spectroscopic companion to $\psi^1$ Dra A from 1991-1995, but found no radial velocity variation. More recently \cite{Toyota2009} noted a linear trend in their radial velocity measurements, and predicted a companion with $M > 50 M_J$. Our data has a much longer time baseline than either of the previous studies, and shows a significant fraction of the orbit which has recently reached quadrature. Furthermore, Brugameyer et al (2015) use adaptive optics imaging to detect a $\sim 4500$ K companion 155 mas from $\psi^1$ Dra A, which they hypothesize is the source of the orbital motion seen in the primary star radial velocity measurements. 

Here, we use all of our spectra of $\psi^1$ Dra A to search directly for the spectral lines of the companion and measure the system mass ratio. We describe the observations and data reduction in Section \ref{sec:obs}, and the method we use to search for the companion in Section \ref{sec:method}. Finally, we estimate the mass-ratio of the system and give the parameters for the companion in Section \ref{sec:orbit}.


\section{Observations and Data Reduction}
\label{sec:obs}
All data were taken at the 2.7m Harlan J. Smith Telescope at McDonald observatory using the 2dcoude echelle spectrograph \citep{TS23} at a resolving power $R\equiv \frac{\lambda}{\Delta \lambda} = 60000$. The starlight was filtered through a temperature-stabilized $I_2$ cell to imprint many sharp absorption lines on each spectrum to use for both a precise velocity metric \citep{Butler1996} and to model the instrument profile \citep{Endl2000}. The raw CCD data were reduced with standard IRAF\footnote{IRAF is distributed by the National Optical Astronomy Observatories, which are operated by the Association of Universities for Research in Astronomy, Inc., under cooperative agreement with the National Science Foundation.} tasks, and include steps for overscan trimming, bad pixel processing, bias frame subtraction, scattered light removal, flat field division, order extraction, and wavelength solution fitting using a Th-Ar calibration lamp spectrum. Particularly strong cosmic ray hits were removed manually by interpolating across nearby pixels.

We used the \emph{Austral} code \citep{Endl2000} to measure the differential radial velocity of $\psi^1$ Dra A at each observation. We provide the velocity measurements after shifting into the system velocity rest frame in Table \ref{tab:rv_data}. The velocity shift necessary to convert from the differential radial velocities to this frame is found in Section \ref{sec:orbit}. Table 1 also gives the measurements of the companion radial velocity (described in the next section).



\begin{deluxetable}{lrrrr}
\tabletypesize{\footnotesize}
\tablewidth{0pt}
%\tablenum{1}
\tablecaption{Observations of $\psi^1$ Dra A \label{tab:rv_data}}
\tablehead{
\colhead{Julian Date} & \colhead{ $v_1$ }  & \colhead{$\sigma_{v_1}$ } & \colhead{ $v_2$ }  & \colhead{$\sigma_{v_2}$ }  \\
 & \colhead{(m/s)}  & \colhead{(m/s)} & \colhead{(m/s)}  & \colhead{(m/s)} }

\startdata

 2451809.66 &  -2.174 &   0.013 &  \nodata &  \nodata \\
 2451809.67 &  -2.172 &   0.014 &  \nodata &  \nodata \\
 2452142.68 &  -2.259 &   0.012 &  \nodata &  \nodata \\
 2453319.64 &  -1.668 &   0.011 &  \nodata &  \nodata \\
 2453585.85 &  -1.542 &   0.010 &  \nodata &  \nodata \\
 2453585.88 &  -1.551 &   0.011 &  \nodata &  \nodata \\
 2453634.64 &  -1.446 &   0.011 &  \nodata &  \nodata \\
 2453635.62 &  -1.547 &   0.009 &  \nodata &  \nodata \\
 2453655.64 &  -1.390 &   0.009 &  \nodata &  \nodata \\
 2453655.64 &  -1.321 &   0.027 &  \nodata &  \nodata \\
 2453689.54 &  -1.436 &   0.008 &  \nodata &  \nodata \\
 2453907.85 &  -1.141 &   0.011 &  \nodata &  \nodata \\
 2453928.80 &  -1.243 &   0.012 &  \nodata &  \nodata \\
 2454019.60 &  -1.171 &   0.012 &  \nodata &  \nodata \\
 2454279.75 &  -1.033 &   0.011 &  \nodata &  \nodata \\
 2454279.76 &  -1.044 &   0.010 &  \nodata &  \nodata \\
 2454309.79 &  -1.080 &   0.013 &  \nodata &  \nodata \\
 2454345.63 &  -0.830 &   0.010 &  \nodata &  \nodata \\
 2454401.56 &  -0.945 &   0.009 &  \nodata &  \nodata \\
 2454662.93 &  -0.752 &   0.015 &     0.36 &     0.37 \\
 2454665.77 &  -0.615 &   0.014 &     0.75 &     0.35 \\
 2454665.77 &  -0.609 &   0.015 &     0.64 &     0.36 \\
 2454730.71 &  -0.644 &   0.014 &     0.68 &     0.36 \\
 2455100.57 &  -0.226 &   0.016 &     0.04 &     0.44 \\
 2455100.58 &  -0.210 &   0.014 &     0.16 &     0.44 \\
 2455398.75 &   0.108 &   0.015 &    -0.89 &     0.51 \\
 2455790.72 &   0.876 &   0.021 &    -2.34 &     0.67 \\
 2455869.58 &   1.111 &   0.017 &    -1.66 &     0.60 \\
 2455910.57 &   1.221 &   0.018 &    -1.82 &     0.63 \\
 2455992.02 &   1.437 &   0.012 &    -2.22 &     0.65 \\
 2456016.93 &   1.558 &   0.014 &    -2.37 &     0.62 \\
 2456106.78 &   1.683 &   0.015 &    -3.24 &     0.67 \\
 2456138.84 &   1.844 &   0.020 &    -3.51 &     0.73 \\
 2456145.65 &   1.846 &   0.025 &    -4.14 &     0.80 \\
 2456145.66 &   1.828 &   0.018 &    -4.19 &     0.79 \\
 2456145.66 &   1.864 &   0.018 &    -3.91 &     0.79 \\
 2456173.73 &   1.854 &   0.018 &    -3.48 &     0.75 \\
 2456401.97 &   2.864 &   0.014 &    -5.87 &     0.69 \\
 
\enddata

%\tablecomments{The velocities given here are shifted into the system velocity rest frame using the results of the analysis presented in Section \ref{sec:orbit}. The raw radial velocity data is available at \url{https://github.com/kgullikson88/Planet-Finder}}
\end{deluxetable}
 


\begin{deluxetable}{lrrrr}
\tabletypesize{\footnotesize}
\tablewidth{0pt}
\tablenum{1}
\tablecaption{Observations of $\psi^1$ Dra A (continued)}
\tablehead{
\colhead{Julian Date} & \colhead{ $v_1$ }  & \colhead{$\sigma_{v_1}$ } & \colhead{ $v_2$ }  & \colhead{$\sigma_{v_2}$ }  \\
 & \colhead{(m/s)}  & \colhead{(m/s)} & \colhead{(m/s)}  & \colhead{(m/s)} }

\startdata

 2456401.97 &   2.841 &   0.012 &    -6.15 &     0.68 \\
 2456433.74 &   3.138 &   0.013 &    -6.14 &     0.62 \\
 2456433.74 &   3.108 &   0.012 &    -6.21 &     0.65 \\
 2456435.87 &   3.108 &   0.015 &    -6.33 &     0.64 \\
 2456435.87 &   3.104 &   0.015 &    -6.68 &     0.64 \\
 2456461.87 &   3.257 &   0.012 &    -6.42 &     0.63 \\
 2456461.88 &   3.250 &   0.015 &    -6.11 &     0.65 \\
 2456461.88 &   3.225 &   0.016 &    -6.06 &     0.61 \\
 2456465.80 &   3.196 &   0.014 &    -6.18 &     0.53 \\
 2456497.86 &   3.473 &   0.019 &    -7.13 &     0.73 \\
 2456519.62 &   3.664 &   0.015 &    -7.93 &     0.59 \\
 2456525.66 &   3.624 &   0.017 &    -7.38 &     0.64 \\
 2456560.58 &   3.711 &   0.013 &    -7.22 &     0.90 \\
 2456564.59 &   3.680 &   0.015 &    -7.05 &     0.86 \\
 2456613.55 &   3.988 &   0.016 &    -6.51 &     0.91 \\
 2456614.58 &   4.038 &   0.012 &    -7.06 &     0.85 \\
 2456755.98 &   5.208 &   0.014 &   -10.84 &     0.74 \\
 2456759.97 &   5.265 &   0.015 &   -11.28 &     0.76 \\
 2456784.84 &   5.502 &   0.017 &   -11.69 &     0.81 \\
 2456816.67 &   5.794 &   0.014 &   -12.17 &     0.74 \\
 2456816.67 &   5.806 &   0.015 &   -12.49 &     0.72 \\
 2456860.73 &   6.301 &   0.016 &   -13.97 &     0.88 \\
 2456860.73 &   6.321 &   0.015 &   -14.11 &     0.83 \\
 2456885.62 &   6.507 &   0.015 &   -15.29 &     0.84 \\
 2456938.63 &   7.089 &   0.016 &   -16.77 &     1.03 \\
 2456938.64 &   7.072 &   0.015 &   -16.45 &     0.96 \\
 2457092.02 &   8.013 &   0.015 &   -18.09 &     1.00 \\
 2457109.85 &   7.976 &   0.015 &   -19.40 &     1.10 \\
 2457118.96 &   7.886 &   0.016 &   -17.82 &     0.90 \\
 2457150.92 &   7.584 &   0.017 &  \nodata &  \nodata \\
 2457174.96 &   7.167 &   0.017 &   -16.13 &     0.81 \\
 2457214.83 &   6.139 &   0.017 &   -13.16 &     0.82 \\
 2457214.84 &   6.152 &   0.016 &   -13.85 &     0.90 \\
 2457216.73 &   6.119 &   0.016 &   -13.66 &     0.86 \\
 2457216.73 &   6.128 &   0.015 &   -13.69 &     0.80 \\
 2457245.60 &   5.201 &   0.016 &   -11.52 &     0.77 \\
 2457245.61 &   5.199 &   0.016 &   -11.16 &     0.72 \\
 2457248.61 &   5.237 &   0.017 &   -11.05 &     0.70 \\
 
\enddata

\tablecomments{The velocities given here are shifted into the system velocity rest frame using the results of the analysis presented in Section \ref{sec:orbit}. The raw radial velocity data is available at \url{https://github.com/kgullikson88/Planet-Finder}}
\end{deluxetable}
 
 
 
 


\section{Companion Search}
\label{sec:method}

We use a cross-correlation analysis inspired by recent work attempting to detect light from planetary companions around late-type stars \citep{Gullikson2013, Martins2013} to search for the companion ($\psi^1$ Dra C). We start by dividing all spectra by the blaze function of the spectrograph, and further divide them by an empirical $I_2$ cell absorption spectrum in the spectral orders with $500 < \lambda < 640$ nm. The blaze function is derived by fitting a high-order polynomial to the extracted spectrum of an incandescent light source (a flat lamp), and the empirical $I_2$ spectrum is the spectrum of a flat lamp with the $I_2$ cell inserted in the light path. Both the flat lamp and $I_2$ spectra are observed each day of each observing run. We use the Telfit code \citep{Gullikson2014} to fit and remove the unsaturated telluric absorption lines in the spectrum, and cross-correlate each residual spectrum against a Phoenix model spectrum \citep{Husser2013} with parameters

\begin{itemize}
\item $T_{\rm eff} = 4400$ K
\item $\log{g} = 4.5$ (cgs units)
\item {[}Fe/H{]} = 0.0
\end{itemize}

The model temperature was chosen based on high contrast imaging in Brugameyer et al (2015), which finds a companion with approximately that temperature. We shift each CCF so that the dominant peak, which signifies the match of the M-star model template with the F-type primary star, falls at $v=0$. This effectively puts the cross-correlation functions in the rest frame of the primary star, although there is a constant velocity offset because the observed spectra are wavelength-calibrated to air wavelengths, while the model spectra is calculated in vacuum wavelengths. We shift the model spectra to air wavelengths using an approximate formula \citep{Ciddor96}, but expect a small difference between the calculated and actual index of refraction since the observatory is at a higher altitude with correspondingly lower air pressure than where the conversion formula was calibrated. The difference in index of refraction translates into a velocity offset, which we denote as $\Delta v_2$ in later sections of this paper.

We normalize each CCF by subtracting a quadratic function that we fit well away from the peak, and then dividing by the height of the CCF at $v=0$ (the peak). The average of these shifted CCFs is a close estimate for the cross-correlation function of the M-star template with the F5 primary star, since the contribution from the companion is diluted by shifting the CCFs to the primary star rest frame. We remove the contribution from the primary star by subtracting the average from each CCF. The result is a series of residual cross-correlation functions that are estimates for the CCF of the companion spectrum against the 4400 K model spectrum template, with significant noise. We show the residual CCFs in Figure \ref{fig:resids}; the trace of the companion star is easily visible as the dark curve near the top middle. We are unable to recover the companion signal at early dates when the two stars were close to each other in velocity space.



\begin{figure}
  \centering
  \includegraphics[width=\columnwidth]{Resid_CCFs.pdf}
  \caption{Cross-correlation functions of a 4400 K model spectrum template with the data, after subtraction of the average CCF. Early dates are on bottom. The dark curve in the top middle is the signal of the companion star.}
  \label{fig:resids}
\end{figure}



\begin{figure}
  \centering
  \includegraphics[width=\columnwidth]{Typical_CCF.pdf}
  \caption{An example of a typical residual cross-correlation function. The dominant peak denotes the template match of the 4400 K template with the companion to $\psi^1$ Draconis A. The velocities are in the approximate rest frame of the primary star (see text for details), and the centroid and FWHM are given as vertical dotted lines.}
  \label{fig:ccf_typical}
\end{figure}



\section{Mass Ratio Estimation}
\label{sec:orbit}

We measure the radial velocity of the companion at each epoch by finding the maximum and full-width-half-maximum (FWHM) of the residual CCF. Based off Figure \ref{fig:resids}, we only use the portion of the CCFs with $-50 < v < 10 \rm km\ s^{-1}$; we show a typical residual CCF in Figure \ref{fig:ccf_typical}. Since the CCFs were shifted to subtract the contribution from the primary star, the measured velocities ($v_{m, 2}$) are related to the true barycentric velocities ($v_1$ and $v_2$ for the primary and secondary, respectively) and the constant shift described above ($\Delta v_2$) through

\begin{equation}
v_{m, 2}(t) = v_2(t) - v_1(t) + \Delta v_2
\label{eqn:vel}
\end{equation}

We give the companion velocities in Table \ref{tab:rv_data}, after converting from $v_{m, 2}$ to $v_2$ using Equation \ref{eqn:vel} and the results of the analysis described below. The uncertainties given in Table \ref{tab:rv_data} are determined from the CCF peak width and the scaling factor ($f$) derived below. Both the primary and secondary velocities given in Table \ref{tab:rv_data} are for the reader's convenience; we use the raw measurements in the orbital fit.

We use the emcee code \citep{emcee} to perform a Markov Chain Monte Carlo (MCMC) fit to an orbital solution. We fit the semi-amplitudes for both the primary and secondary stars (K1 and K2, respectively), the longitude of pericenter ($\omega$), the eccentricity (e), and the periastron passage epoch ($T_0$). In addition to the orbital parameters, we fit a velocity offset for both the primary and secondary star measurements ($\Delta v_1$ and $\Delta v_2$, respectively). We need to fit two velocity offsets because neither measurement is an absolute radial velocity measurement, and because they were measured in very different ways. We also fit a scale factor to apply to the uncertainties in the secondary star velocity measurements, which is necessary because the FWHM of the peak over-estimates the uncertainty in the peak velocity (see Figure \ref{fig:ccf_typical}), but should still be proportional to it (i.e. a broader peak has a higher uncertainty than a narrower peak). Finally, we account for radial velocity variations caused by instrumental or astrophysical noise by fitting an "RV jitter" term, which we add in quadrature to the primary star uncertainties. This term acts to increase the uncertainties on most other parameters, and so makes our estimates more conservative. Since the measured primary star velocities are differential and the measured secondary star velocities are measured in the rest frame of the primary star, we cannot measure the system radial velocity. 

 We give the median value and uncertainty for each parameter in Table \ref{tab:orbit}. The uncertainties are estimated from the posterior probability distribution samples such that the lower and upper bounds give the 16th and 84th percentile (i.e. they are $1\sigma$ confidence intervals). We plot the best-fit orbit with the data in Figure \ref{fig:orbit}, with the uncertainties on the companion velocities scaled. We show the posterior probability distributions for the orbital parameters in Figure \ref{fig:orbit_dist} and for the data shifting and scaling parameters ($\Delta v_1$, $\Delta v_2$, and the uncertainty scaling) in Figure \ref{fig:datamanip_dist}.

Next, we calculate a series of derived quantities to characterize the companion and report them in Table \ref{tab:orbit}. The mass ratio of the system is the ratio $K1/K2 = 0.47$. We estimate the primary star mass by interpolating Dartmouth isochrones \citep{Dotter2008} with the `isochrones' code \citep[described in][]{Montet2015}, and using spectroscopic parameters derived in Brugameyer et al (2015). The secondary mass is $M_2 = qM_1 \sim 0.70 M_{\odot}$; assuming the same age and metallicity as the primary, the Dartmouth isochrones give an expected temperature of $\sim 4400 $ K. This temperature is in excellent agreement with the high contrast imaging data, which supports a companion of $\sim 4400$ K with large uncertainties. With both the primary and secondary star mass, we calculate the orbital inclination and semimajor axis and report them in Table \ref{tab:orbit}.



\begin{deluxetable}{rl}
\tabletypesize{\footnotesize}
\tablewidth{0pt}
\tablenum{2}
\tablecaption{ Orbital parameters for the $\psi^1$ Draconis A subsystem. }
%\tablehead{
%\colhead{Component} & \colhead{$T_{\rm eff}$} & \colhead{$\log{g}$} & \colhead{{[}Fe/H{]}}& \colhead{$M (M_{\odot})$} }

\startdata
\cutinhead{Parameters from Brugameyer et al (2015)}
$T_{\rm eff,1}$ (K) & $6544 \pm 42$ \\
$\log{g}$ & $3.90 \pm 0.11$ \\
{[}Fe/H{]} & $-0.10 \pm 0.05$ \\

\cutinhead{Parameters derived in this work}
K1 ($\rm km\ s^{-1}$) & $5.18^{+0.04}_{-0.03}$ \\
K2 ($\rm km\ s^{-1}$) & $11.1 \pm 0.2$ \\
Period (days) & $6774^{+271}_{-167}$ \\
Periastron passage time (JD) & $2450388{+169}_{-273}$ \\
$\omega$ (degrees) & $32.6 \pm 0.7$ \\
e & $0.679^{+0.006}_{-0.004}$ \\
$\Delta v_1\ (\rm km\ s^{-1})$ & $4.10^{+0.06}_{-0.09}$ \\
$\Delta v_2\ (\rm km\ s^{-1})$ & $-5.4^{+0.3}_{-0.2}$ \\
Companion uncertainty scale factor ($f$) & $0.17 \pm 0.02 $\\
RV jitter (m s$^{-1}$) & $72^{+7}_{-5}$ \\
reduced $\chi^2$ & 0.41 \\ \\
$q$ & $0.466 \pm 0.008$ \\
$M_1$ (M$_{\odot}$) & $1.38^{+0.15}_{-0.08}$ \\
$M_2$ (M$_{\odot}$) & $0.70 \pm 0.07$ \\
$T_{\rm eff,2}$ (K) & $4400 \pm 300$ \\
i (degrees) & $31 \pm 1$ \\
a (AU) & $9.1^{+0.4}_{-0.3}$ \\


\enddata
\tablecomments{The primary mass is derived using the spectroscopic $T_{\rm eff}$, $\log{g}$, and [Fe/H] and interpolating Dartmouth isochrones. The companion temperature is likewise derived from the companion mass using Dartmouth isochrones of the same metallicity.}
\label{tab:orbit}
\end{deluxetable}




\begin{figure}

  \includegraphics[width=\columnwidth]{SB2_Orbit.pdf}
  \caption{Best-fit double-lined orbit for the $\psi^1$ Draconis AC subsystem. There are no measurements of the companion at early dates because they could not be reliably measured in the residual cross-correlation functions.}
  \label{fig:orbit}
  
\end{figure}



\section{Discussion and Conclusions}

We use nearly 15 years of time-series spectra of the star $\psi^1$ Draconis A to search for the spectral lines of a companion identified by a large amplitude trend in the primary star radial velocities and later by direct imaging. We cross-correlate each spectrum against a Phoenix model spectrum of an $4400$ K star and subtract the average CCF. The residual CCFs clearly show the template match with the companion (Figure \ref{fig:resids}), and we are able to measure the companion radial velocities for most dates. 

We use the radial velocity measurements for both the primary and secondary stars to find an orbital solution to the now double-lined spectroscopic binary. The summary values and posterior probability distributions of the fitted parameters are given in Table \ref{tab:orbit} and Figures \ref{fig:orbit_dist} and \ref{fig:datamanip_dist}. Finally, we report the mass and expected temperature of the companion as well as the orbital inclination and semi-major axis. The temperature agrees well with high contrast imaging, validating our method.

The $\psi^1$ Draconis system is therefore a hierarchical multiple system with the component parameters given in Table \ref{tab:orbit}. $\psi^1$ Dra A and B are separated by $\sim 680$ AU and have a mass-ratio $q = 0.82$, while A and C (the new companion) have a much closer orbit with with $a = 9.1$ AU and $q = 0.47$. 

This method could be used to search for the spectral lines of stellar companions to other stars observed with high precision radial velocity surveys. To that end, and in the goal of open science, we make the source code used for the analysis and generating the plots for this paper available at \url{https://github.com/kgullikson88/Planet-Finder}. The raw radial velocity measurements for both the primary and secondary star, as well as the the MCMC chains, are available at the same url.

This research has made use of the SIMBAD database, operated at CDS, Strasbourg, France, and of Astropy, a community-developed core Python package for Astronomy (Astropy Collaboration, 2013).
It was supported by a start-up grant to Adam Kraus from the University of Texas.



\begin{figure}
  \centering
  \includegraphics[width=\columnwidth]{Distributions_1.pdf}
  \caption{MCMC samples from the posterior distributions for the orbital parameters.}
  \label{fig:orbit_dist}
\end{figure}


\begin{figure}
  \centering
  \includegraphics[width=\columnwidth]{Distributions_2.pdf}
  \caption{MCMC samples from the posterior distributions for the data shift and scaling parameters.}
  \label{fig:datamanip_dist}
\end{figure}


\newpage
\bibliography{}
\end{document}
